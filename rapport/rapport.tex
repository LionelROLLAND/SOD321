\documentclass[12pt,a4paper]{article}
\usepackage[utf8]{inputenc}
\usepackage[french]{babel}
\usepackage[T1]{fontenc}

%\voffset=-1.5cm
%\hoffset=0cm
%\textwidth=16cm
%\textheight=22.0cm

\usepackage{amsmath}
\usepackage{amssymb}

%\setlength{\marginparwidth}{0pt}

\usepackage{subcaption}
\usepackage{float}


\usepackage{enumitem}
\usepackage{hyperref}

\usepackage{tikz}
\usepackage{pgfplots}
\pgfplotsset{compat=newest}
\usepgfplotslibrary{groupplots}
\usepgfplotslibrary{dateplot}
\usepgfplotslibrary{statistics}
\usetikzlibrary{pgfplots.statistics} % LATEX and plain TEX

\usepackage{titlesec}

\usepackage{datetime}

\usepackage{array}
\usepackage{csquotes}

\title{Projet aérodromes}
\author{Antonin Rogé \& Lionel Rolland}
\date{\monthname \ \the\day\textsuperscript{th} \the\year}

\begin{document}

\maketitle

\newpage

\tableofcontents

\newpage

\section{Notations}

Notons $R \subset \mathbb{N}^*$ l'ensemble des régions.
Notons $N$ l'ensemble des aérodromes. $\forall i \in N$, notons $r_i \in R \cup \{0\}$ la région
à laquelle appartient l'aérodrome, $x_i$ sa coordonnée en $x$ et $y_i$ sa
coordonnée en $y$. $x_ij \in \{0; 1\}$. Notons $s$ l'aérodrome de départ et $t$ celui d'arrivée.
Soit $A \subset N \times N$ l'ensemble des arcs. $\forall (i,j) \in A$ soit
$a_{ij}$ la variable associée à la sélection de l'arc $(i, j)$.

\section{Contraintes supplémentaires afin de passer dans toutes les régions}

Un set de $|R|$ contraintes simples permettant d'assurer que l'on passe dans toutes les régions :

$$\forall r \in R \quad \sum_{\substack{i \in N \\ r_i = r}}
\Big(\sum_{j \in N \\ (i, j) \in A} a_{i, j} + \sum_{j \in N \\ (j, i) \in A} a_{j, i} \Big) \geq 1$$

Cependant, cette contrainte peut être raffinée pour obtenir des bornes de relaxation PL plus intéressantes.
Supposons sans perte de génaralité que le graphe est connexe, et que $r_s = r_t = 0$.

Dans le cas d'un graphe non orienté, on peut simplifier ces contraintes par :

$$\forall r \in R \quad \sum_{\substack{i \in N \\ r_i = r}} \sum_{\substack{j \in N \\ (i, j) \in A}} a_{i, j} \geq 1$$

On peut même encore aller plus loin avec simplement :

$$\forall r \in R \quad \sum_{\substack{i \in N \\ r_i = r}} \sum_{\substack{j \in N \\ r_j \neq r \\ (i, j) \in A}} a_{i, j} \geq 1$$

\end{document}